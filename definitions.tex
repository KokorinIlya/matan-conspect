\documentclass[paper=a4, fontsize=15pt]{article}

\usepackage[russian]{babel}
\usepackage{scrextend}
\usepackage[utf8x]{inputenc}
\usepackage[T1,T2A]{fontenc}
\usepackage[left=1.5cm,right=1.5cm,top=1.5cm,bottom=1.5cm,bindingoffset=0cm]{geometry}
\usepackage[pdftex]{graphicx}
\usepackage{amsmath}
\usepackage{mathtools}
\usepackage{ulem}
\usepackage{mathrsfs}
\usepackage{amsfonts}
\usepackage{dsfont}
\usepackage{amssymb}
\usepackage{cmap}
\usepackage{hyperref}


\parindent=0cm

\title{Определения по матану, семестр 4}

\begin{document}
\maketitle
\tableofcontents
\newpage

\section{Свойство, выполняющееся почти везде}
$ (X,\mathds{A},\mu)$  - пространство с мерой, и $\omega (x)$  -- утверждение, зависящее от точки $x$.

$E := \{x: \omega(x) $ --- ложно\} и $\mu E$ = 0. Тогда говорят, что $\omega (x)$ верно при почти всех (п.в.) $x$.

\section{Сходимость почти везде}
$ (X,\mathds{A},\mu)$  - пространство с мерой, и $f_n, f: X \rightarrow \overline{\mathds{R}}.$

Говорим, что $f_n \rightarrow f(x)$ почти везде, если $\{x: f_n(x) \not \rightarrow f(x)\}$ измеримо и имеет меру $0$.

\section{Сходимость по мере}
$(X, a, \mu)$ - пространство с мерой, $\mu \cdot X < +\infty$ \\ 
$f_n , f : X \rightarrow \overline R$ - п.в. конечны\\
Говорят, что $f_n$ сходится к $f$ по мере $\mu$ (при $n \rightarrow +\infty$) (обозначается $f_n\stackrel{\mu}{\Rightarrow}f$) если $\forall\epsilon > 0$ $\mu(X(|f_n - f| > \epsilon))\stackrel{n\rightarrow+\infty}{\rightarrow} 0$

\section{Теорема Егорова о сходимости почти везде и почти равномерной сходимости}
$(X, a, \mu)$ - пространство с мерой\\ 
$f_n , f : X \rightarrow R$ - п.в. конечны, измеримы \\
$f_n \rightarrow f$.\\
Тогда эта сходимость ``почти равномерная''

\section{Интеграл ступенчатой функции}
$<\mathds{X}, \mathds{A}, \mu>$ - пространство с мерой

$f = \sum\limits_{k = 1}^{n}(\lambda_k \cdot \chi_{E_k})$ - ступенчатая функция, $E_k$ - измеримые дизъюнктные множества, $f \geqslant 0$

Интегралом ступенчатой функции $f$ на множестве $\mathds{X}$ назовём $$\int\limits_\mathds{X} f d\mu := \sum\limits_{k = 0}^{n} \lambda_k \cdot \mu E_k$$

Будем считать, что $[0 \cdot \infty = 0]$

\section{Интеграл неотрицательной измеримой функции}
$<\mathds{X}, \mathds{A}, \mu>$ - пространство с мерой

$f$ - измеримо, $f \geqslant 0$, её интегралом на множестве $\mathds{X}$ назовём

$$\int\limits_{\mathds{X}} f d\mu := sup (\int\limits_{\mathds{X}} g)$$

, где $0 \leqslant g \leqslant f, g - $ступенчатая

\section{Суммируемая функция}
$<\mathds{X}, \mathds{A}, \mu>$ - пространство с мерой

$f - $измерима, $\int\limits_{\mathds{X}}f^+$ или $\int\limits_{\mathds{X}}f^-$ конечен (хотя бы один из них).

Тогда интегралом $f$ на $\mathds{X}$ назовём $$\int\limits_{\mathds{X}}f d\mu := \int\limits_{\mathds{X}}f^+ - \int\limits_{\mathds{X}}f^+$$

Тогда если конечен $\int\limits_{\mathds{X}} f,$ (то есть конечны интегралы по обеим срезкам), то $f$ называют суммируемой

\section{Интеграл суммируемой функции}
$<\mathds{X}, \mathds{A}, \mu>$ - пространство с мерой

$f -$ измерима, $E \in \mathds{A}$

Тогда интегралом $f$ на множестве $E$ назовём

$$\int\limits_{\mathds{E}}f d\mu := \int\limits_{\mathds{X}}f \cdot \chi(E) d\mu$$

$f$ суммируемая на $E$, если $\int\limits_{\mathds{X}}f^+ \chi(E)$ и $\int\limits_{\mathds{X}}f^- \chi(E)$ конечны

\end{document}