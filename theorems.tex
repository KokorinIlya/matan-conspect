\documentclass[paper=a4, fontsize=13.2pt]{article}

\usepackage[russian]{babel}
\usepackage{scrextend}
\usepackage[utf8x]{inputenc}
\usepackage[T1,T2A]{fontenc}
\usepackage[left=1.5cm,right=1.5cm,top=1.5cm,bottom=1.5cm,bindingoffset=0cm]{geometry}
\usepackage[pdftex]{graphicx}
\usepackage{amsmath}
\usepackage{mathtools}
\usepackage{ulem}
\usepackage{mathrsfs}
\usepackage{amsfonts}
\usepackage{dsfont}
\usepackage{amssymb}
\usepackage{cmap}
\usepackage{hyperref}


\parindent=0cm

\title{Теоремы по матану, семестр 4}

\begin{document}
\maketitle
\tableofcontents
\newpage

\section{Характеризация измеримых функций с помощью ступенчатых (формулировка). Следствия}
$(X,\mathds{A},\mu)$~--- пространство с мерой.

$f$~--- измеримая функция на $X$, $\forall x\ f(x) \geq 0$. Тогда $\exists$ ступенчатые функции $f_n$, такие что:
\begin{enumerate}
    \item $\forall x$ $0 \leq f_n(x) \leq f_{n+1}(x) \leq f(x)$.
    \item $f_n(x)$ поточечно сходится к $f(x)$.
\end{enumerate}

Следствие $1$:

$f: X \rightarrow \overline {\mathds{R}}$ измеримая. Тогда $\exists$ ступенчатая $f_n: \forall x:  lim f_n(x) = f(x)$ и $|f_n(x)| \leq |f(x)|$.

\emph{Доказательство:}

\begin{enumerate}
	\item Рассмотрим $f = f^+ - f^-. f^+ = max(f, 0), f^- = max(-f, 0)$. Срезки измеримы: $E(f^+  < a) = E(f < a) \cap E(0 < a)$, при этом $f$  и $g \equiv 0$ измеримы ($f^-$ измерима аналогично).
	\item Срезки измеримы и неотрицательны, тогда по теореме существуют ступенчатые функции $f^+_n \rightarrow f^+, f^-_n \rightarrow f^-$. Тогда и $f^+_n - f^-_n$ это ступенчатая функция, при этом по свойству пределов: $f^+_n - f^-_n \rightarrow f^+ - f^- = f$
	
	(почему верно с модулем -- непонятно, если в лоб, то неверно. Спрошу 19.02)
\end{enumerate}

Следствие $2$:

$f, g$ --- измеримые функции. Тогда $fg$ -- измеримая функция. При этом считаем, что $0 \cdot \infty = 0$.

\emph{Доказательство:}
\begin{enumerate}
	\item Рассмотрим $f_n \rightarrow f: |f_n| \leq |f|, g_n \rightarrow g: |g_n| \leq |g|$ из первого следствия. Тогда $f_ng_n \rightarrow fg$

	(и что с того? У нас же другое определение измеримых функций. Спрошу 19.02)
\end{enumerate}

Следствие $3$:

$f, g$ --- измеримые функции. Тогда $f + g$ -- измеримая функция. При этом считаем, что $\forall x$ не может быть, что $f(x) = \pm \infty, g(x) = \mp \infty$

\emph{Доказательство:}

Доказывается как следствие $2$.

\section{Измеримость монотонной функции}

Пусть $E \subset R^m$~--- измеримое по Лебегу, $E' \subset E, \lambda_m (E \setminus E') = 0, f: E \rightarrow \mathds{R}.$ Пусть сужение $f: E' \rightarrow R$ непрерывно. Тогда $f$ измерима на $E$.

\emph{Доказательство:}

\begin{enumerate}
	\item $E(f < a) = E'(f < a) \cup e(f < a), e:=E \setminus E', \lambda_m(e) = 0$. 
	\item $ E'(f < a)$ открыто в $E'$, так как f непрерывна. Поэтому $E' = G \cap E' \Rightarrow$, где $G$ ~--- открытое в $E$ множество.  Значит, $E'(f<a)$ ~--- измеримо по Лебегу, так как оно является борелевским. 
	\item Но и $e(f<a)$ измеримо, так $\lambda_m(e) = 0$, следовательно $E(f < a)$ измеримо как объединение измеримых множеств
\end{enumerate}

Следствие:

$f: <a, b> \rightarrow \mathds{R}$ монотонна. Тогда $f$ измерима.

\emph{Доказательство:}

Множество разрывов монотонной функции НБЧС множество, поэтому можно воспользоваться доказанной теоремой.

\section{Теорема Лебега о сходимости почти везде и сходимости по мере}

\section{Теорема Рисса о сходимости по мере и сходимости почти везде}

\section{Простейшие свойства интеграла Лебега}
\subsection{Для определения (5)}
\begin{enumerate}
	\item $\int\limits_{\mathds{X}}f$ не зависит от представления $f$ как ступенчатой функции, то есть если $f$ реализуется как $f = \sum\limits_{k}(\lambda_k \cdot \chi_{E_k})$ и как $f = \sum\limits_{l}(\alpha_l \cdot \chi_{G_l})$, интегралы по этим функциям равны
	
	\emph{Доказательство:}
	
	Выпишем общее разбиение для этих двух разбиений
	
	Пусть $F_{ij} = E_i \cap G_j$
	
	Тогда $f = \sum\limits_{k}(\lambda_k \cdot \chi_{E_k}) = \sum\limits_{l}(\alpha_l \cdot \chi_{G_l}) = \sum\limits_{i, j}(\lambda_i (= \alpha_j) \cdot \chi_{F_{i, j}})$
	
	$\int f = \sum\limits_{i, j}(\lambda_i \cdot \mu F_{i, j}) = \sum\limits_i (\lambda_i \cdot \sum\limits_j (\mu F_{i, j})) = \sum\limits_i (\lambda_i \cdot \mu E_i) = \int f$ для первого разбиения
	
	Аналогично для второго разбиения получаем 
	
	$\int f = \sum\limits_{i, j}(\lambda_i \cdot \mu F_{i, j}) = \sum\limits_j (\alpha_i \cdot \sum\limits_i (\mu F_{i, j})) = \sum\limits_j (\lambda_j \cdot \mu G_i) = \int f$ для второго разбиения, что и требовалось доказать
	
	\item $f, g$ -измеримые ступенчатые функции, $f \leqslant g$, тогда $\int\limits_{\mathds{X}} f \leqslant \int\limits_{\mathds{X}} g$
	
	\emph{Доказательство:}
	
	Пусть $f = \sum\limits_{k}(\lambda_k \cdot \chi_{E_k})$, $g = \sum\limits_{l}(\alpha_l \cdot \chi_{G_l})$
	
	Аналогично доказательству предыдущей теоремы, строим общее ступенчатое разбиение
	
	Пусть $F_{ij} = E_i \cap G_j$
	
	Тогда $\int f = \sum\limits_{i, j}(\lambda_i \cdot \mu F_{i, j}) \leqslant \sum\limits_j(\alpha_j \cdot \mu F_{i, j}) = \int g$, что и требовалось доказать
\end{enumerate}

\subsection{Для окончательного определения}

\begin{enumerate}
	\item Монотонность
	$f \leqslant g \Rightarrow \int\limits_{\mathds{X}} f \leqslant \int\limits_{\mathds{X}} g$
	
	\emph{Доказательство:}
	\begin{enumerate}
		\item $f, g \geqslant 0$, тогда доказательство тривиально (по свойствам супремума)
		\item $\int\limits_{\mathds{X}} f = \int\limits_{\mathds{X}} f^+ - \int\limits_{\mathds{X}} f^-$
		
		$\int\limits_{\mathds{X}} g = \int\limits_{\mathds{X}} g^+ - \int\limits_{\mathds{X}} g^-$
		
		Из того, что $\int\limits_{\mathds{X}} f^+ \leqslant \int\limits_{\mathds{X}} g^+$, а $\int\limits_{\mathds{X}} f^- \geqslant \int\limits_{\mathds{X}} g^-$ следует, что $\int\limits_{\mathds{X}} f \leqslant \int\limits_{\mathds{X}} g$
	\end{enumerate}

	\item 
	$\int\limits_{\mathds{E}} 1 \cdot d \mu = \mu E$
	
	$\int\limits_{\mathds{E}} 0 \cdot d \mu = 0$
	
	Очевидно из определения интеграла ступенчатой функции
	
	\item $\mu E = 0, f $-измерима, тогда $\int\limits_{\mathds{E}}f = 0$, даже если $f = \infty$ на $\mathds{E}$
	
	\emph{Доказательство:}
	
	\begin{enumerate}
		\item $f $-ступенчатая $\Rightarrow$ ограниченная
		
		$f = \sum\limits_{k = 1}^{n}(\lambda_k \cdot \chi_{E_k})$, тогда $\int\limits_\mathds{E} f = \sum \lambda_k \cdot \mu (E \cap E_k)$
		
		Но $\mu (E \cap E_k) = 0$ (так как $\mu E = 0$), тогда $\int\limits_\mathds{E} f = 0$
		
		\item $f$ - измеримая, $f \geqslant 0$. 
		
		$\int\limits_\mathds{E} f = \sup (\int\limits_\mathds{E} g)$, где $0 \leqslant g \leqslant f$, $g$ - ступенчатая
		
		Тогда $\int\limits_\mathds{E} f = \sup (0) = 0$
		
		\item 
		$f$ - произвольная измеримая
		
		Тогда $\int\limits_\mathds{E} f = \int\limits_{\mathds{E}} f^+ - \int\limits_{\mathds{E}} f^- = 0 - 0 = 0$
	\end{enumerate}

	\item  
	\begin{enumerate}
		\item
		$\int\limits_{\mathds{E}} -f = - \int\limits_{\mathds{E}} f$
		
		\item
		$\forall c \in \mathds{R}: \int\limits_{\mathds{E}} (c \cdot f) = c \cdot \int\limits_{\mathds{E}} f $
	\end{enumerate}
	
	\emph{Доказательство:}
	\begin{enumerate}
		\item
		$(-f)^+ = f^-$
		
		$(-f)^- = f^+$
		
		Тогда $\int\limits_{\mathds{E}} -f = \int\limits_{\mathds{E}} (-f)^+ - \int\limits_{\mathds{E}} (-f)^- = \int\limits_{\mathds{E}} f^- - \int\limits_{\mathds{E}} f^+ = -\int\limits_{\mathds{E}} f$
		
		\item
		
		Пусть $c > 0$. Если $c < 0$, то по предыдущему случаю можем рассматривать для $- c < 0$. Если $c = 0$, то по предыдущей теореме $\int\limits_{\mathds{E}} (0 \cdot f) = \int\limits_{\mathds{E}} 0 = 0 = 0 \cdot \int\limits_{\mathds{E}} f$
		
		\begin{enumerate}
			\item
			Пусть $f \geqslant 0$
			
			$\int\limits_{\mathds{E}} (c \cdot f) = \sup (\int\limits_{\mathds{E}} g)$, где $0 \leqslant g \leqslant c \cdot f$, $g$ - ступенчатая
			
			Пусть $g = c \cdot \widetilde{g}$, тогда $\int\limits_{\mathds{E}} (c \cdot f) = \sup (\int\limits_{\mathds{E}} (c \cdot \widetilde{g}))$, где $0 \leqslant c \cdot \widetilde{g} \leqslant c \cdot f$, $\widetilde{g}$ - ступенчатая
			
			Тогда $\int\limits_{\mathds{E}} (c \cdot f) = \sup (\int\limits_{\mathds{E}} (c \cdot \widetilde{g})) = \sup (c \cdot \int\limits_{\mathds{E}} \widetilde{g}) = c \cdot \sup (\int\limits_{\mathds{E}} \widetilde{g}) = c \cdot \int\limits_{\mathds{E}} f $
			
			\item Если $f$ - произвольная:
			
			$\int\limits_{\mathds{E}} (c \cdot f) = \int\limits_{\mathds{E}} (c \cdot f)^+ - \int\limits_{\mathds{E}} (c \cdot f)^- = \int\limits_{\mathds{E}} c \cdot f^+ - \int\limits_{\mathds{E}} c \cdot f^- = c \cdot \int\limits_{\mathds{E}} f^+ - c \cdot \int\limits_{\mathds{E}} f^- = c \cdot (\int\limits_{\mathds{E}} f^+ - \int\limits_{\mathds{E}} f^-) = c \cdot \int\limits_{\mathds{E}} f$
		\end{enumerate}	
	\end{enumerate}
	
	\item Если существует $\int\limits_{\mathds{E}} f d\mu$, то $|\int\limits_{\mathds{E}} f| \leqslant \int\limits_{\mathds{E}} |f|$
	
	\emph{Доказательство:}
	
	$-|f| \leqslant f \leqslant |f|$
	
	$\int\limits_{\mathds{E}} -|f| \leqslant \int\limits_{\mathds{E}} f \leqslant \int\limits_{\mathds{E}} |f|$
	
	$-\int\limits_{\mathds{E}} |f| \leqslant \int\limits_{\mathds{E}} f \leqslant \int\limits_{\mathds{E}} |f|$
	
	Тогда $|\int\limits_{\mathds{E}} f| \leqslant \int\limits_{\mathds{E}} |f|$
	
	
	\item $f$ - измеримая на $\mathds{E}$, $\mu \mathds{E} < \infty$
	
	$a \leqslant f \leqslant b$, тогда $a \cdot \mu E \leqslant \int\limits_{\mathds{E}} f \leqslant b \cdot \mu E$
	
	\emph{Доказательство: }
	
	$a \leqslant f \leqslant b \Rightarrow \int\limits_{\mathds{E}}a \leqslant \int\limits_{\mathds{E}} f \leqslant \int\limits_{\mathds{E}} b$
	
	$a \cdot \int\limits_{\mathds{E}} 1 \leqslant \int\limits_{\mathds{E}} f \leqslant b \cdot \int\limits_{\mathds{E}} 1$
	
	$a \cdot \mu \mathds{E} \leqslant \int\limits_{\mathds{E}} f \leqslant b \cdot \mu \mathds{E}$
	
	\emph{Следствие:}
	
	Если $f$ - Измеримая и ограниченная на $\mathds{E}, \mu \mathds{E} < \infty$, тогда $f$ - суммируемая на $\mathds{E}$
	
	\item $f$ - суммируемая на $\mathds{E} \Rightarrow f$ почти везде конечная на $\mathds{E}$ (то есть $f \in \alpha^0(\mathds{E})$)
	
	\emph{Доказательство:}
	\begin{enumerate}
		\item Пусть $f \geqslant 0$
		
		Пусть $f = +\infty$ на $A$ и пусть $\mu A > 0$
		
		Тогда $\forall n \in \mathds{N}: f \geqslant n \cdot \chi_A$
		
		Тогда $\forall n \in \mathds{N}: \int\limits_{\mathds{E}} f \geqslant n \cdot \int\limits_{\mathds{E}} \chi_A = n \cdot \mu A \Rightarrow \int\limits_{\mathds{E}} f = + \infty$
		
		\item $f$ любого знака
		
		Распишем $f = f^+ - f^-$, по предыдущему пункту $f^+, f^-$ конечны почти везде $\Rightarrow f$ тоже конечно почти везде 
	\end{enumerate}
	
\end{enumerate}

\section{Счетная аддитивность интеграла (по множеству)}
$(X,\mathds{A},\mu)$~--- пространство с мерой, $A = \bigsqcup\limits_{i=1}^{\infty}A_{i} -$ измеримы. $f: X \rightarrow \mathbb{\overline{R}} - $ изм., $f \geqslant 0$

\emph{Тогда:} ${\displaystyle \int\limits_{A}f = \sum\limits_{i=1}^{\infty} \int\limits_{A_{i}}f}$

\emph{Доказательсво:}

\begin{enumerate}
	\item Для начала докажем это для ступенчатых функций. Пусть $f = \sum\limits_{k} (\lambda_k \cdot \chi_{E_k})$
	
 $\int\limits_{A}fd\mu = \sum\limits_{k} (\lambda_k \cdot \mu(E_k \cap A)) =
 \sum\limits_{k} (\lambda_k \cdot (\sum\limits_{i} \mu(E_k \cap A_i))) =
 \sum\limits_{i}(\sum\limits_{k}(\lambda_k \cdot \mu(E_k \cap A_i))) = \sum\limits_{i}(\int\limits_{A_i}f)$
 
	\item Докажем, что $\int\limits_{A}f \leqslant \sum\limits_{i} \int\limits_{A_{i}}f$
	
	\begin{enumerate}
		\item Рассмотрим $0 \leqslant g \leqslant f - $ ступенчатая. $\int\limits_{A}g = \sum\limits_{i} \int\limits_{A_i}g \leqslant \sum\limits_{i} \int\limits_{A_{i}}f$
		 
		 \item Переходя к $sup$ получаем желаемое
	\end{enumerate} 
	
	\item Теперь докажем, что $\int\limits_{A}f \geqslant \sum\limits_{i} \int\limits_{A_{i}}f$ 
	\begin{enumerate}
		\item $A = A_1\sqcup A_2$

		\begin{enumerate}
			\item Рассмотрим $g_1, g_2\ -$ ступенчатые такие, что $0 \leqslant g_i \leqslant f \cdot \chi_{A_i}$
			
			\item Рассмотрим их общее разбиение $E_k:\ g_i = \sum\limits_k (\lambda_k^i \cdot \chi_{E_k})$
			
			\item $g_1 + g_2\ - $ ступенчатая и $0 \leqslant g_1 + g_2 \leqslant f \cdot \chi_{A}$
			
			\item $\int\limits_{A_1}g_1 + \int\limits_{A_2}g_2 \stackrel{lemma}{=} \int\limits_{A}(g_1 + g_2) \stackrel{iii}{\leqslant} \int\limits_{A}f$
			
			\item Поочерёдно переходя к $sup$ по $g_1$ и $g_2$ получаем: $\int\limits_{A_1}f + \int\limits_{A_2}f \leqslant \int\limits_{A}f$ 		
		\end{enumerate}
		
	\item $\forall n \in \mathbb{N}$, что $A = \bigsqcup\limits_{i=1}^{n}A_{i}$ будем последовательно отщеплять последнее множество по $(a)$ 
	
	\item $A = \bigsqcup\limits_{i = 1}^{\infty}A_{i}$
		\begin{enumerate}
			\item Фиксрируем $n \in \mathbb{N}$
			
			\item $A = (\bigsqcup\limits_{i=1}^{n}A_{i}) \sqcup B$, где $B = \bigsqcup\limits_{i=n+1}^{\infty}A_{i}$
			
			\item $\int\limits_{A}f \geqslant \sum\limits_{i=1}^{n} \int\limits_{A_i}f + \int\limits_{B}f \geqslant \sum\limits_{i=1}^{n} \int\limits_{A_i}f$
			
			\item Переходим к $lim$ по $n$
		\end{enumerate}		
		
	\end{enumerate}
	
\end{enumerate}

\emph{Следсвие 1:}
$\ 0 \leqslant f \leqslant g$ - измeримы и  $A \subset B$ - измеримы $\Rightarrow \int\limits_{A}f \leqslant \int\limits_{B}g$

$\int\limits_{B}g \geqslant \int\limits_{B}f = \int\limits_{A}f + \int\limits_{B \setminus A}f \geqslant \int\limits_{A}f$ 

\bigskip

\emph{Следствие 2:}
$f$ - суммируема на $A \Rightarrow \int\limits_{A}f = \sum\limits_{i} \int\limits_{A_{i}}f$ 

Достаточно рассмотреть срезки $f^+$ и $f^-$ 

\bigskip

\emph{Следствие 3:}
$f \geqslant 0$ - изм. $\delta: \mathbb{A} \rightarrow \mathbb{\overline{R}}(A\longmapsto \int\limits_{A}fd\mu) \Rightarrow \delta$ - мера

\section{Теорема Леви}
$(X,\mathds{A},\mu),\ f_n \geqslant 0$ - изм.
 
$f_1(x) \leqslant ...\leqslant f_n(x) \leqslant f_{n+1}(x) \leqslant ...$ при почти всех $x$

$f(x) = \lim\limits_{n \rightarrow \infty}f_n(x)$ при почти всех $x$ (считаем, что при остальных $x: f \equiv 0$)
\\

\emph{Тогда:} $\lim\limits_{n \rightarrow \infty} \int\limits_{X}f_n(x)d\mu = \int\limits_{X}f(x)d\mu$
\\

\emph{Доказательство:}

$N.B. \int\limits_{X}f_n \leqslant \int\limits_{X}f_{n+1} \Rightarrow \exists \lim$

$f$ - измерима как предел последователтности измеримых функций

\begin{enumerate}
	\item $\leqslant$
	
	Очевидно $f_n \leqslant f$ при п.в $x \Rightarrow \int\limits_{X}f_n \leqslant \int\limits_{X}f$. Делаем предельный переход по $n$
	
	\item $\geqslant$
		\begin{enumerate}
			\item Логичная редукция: $\lim\limits_{n \rightarrow \infty} \int\limits_{X}f_n(x) \geqslant \int\limits_{x}g$, где $0 \leqslant g \leqslant f$ - ступенчатая
			
			\item Наглая редукция: $\forall c \in (0,1): \lim\int\limits_{X}f_n(x) \geqslant c \cdot \int\limits_{X}g$
				\begin{enumerate}
					\item $E_n = \{x\ |\ f_n(x) \geqslant c \cdot g\}$. Очевидно $E_1 \subset ... \subset E_n \subset E_{n + 1} \subset ...$
					
					\item $\bigcup\limits_{n=1}^{\infty}E_n = X$ т.к. $c < 1$
					
					\item $\int\limits_{X}f_n \geqslant \int\limits_{E_n}f_n \geqslant \int\limits_{E_n}g \Rightarrow \lim \int\limits_{X}f_n \geqslant c \cdot \lim \int\limits_{E_n}g = c \cdot \int\limits_{X}g$
					
					\item Последний знак равно обусловлен тем, что интеграл неотрицательной и измеримой функции по множеству - мера (см. следствие 3 предыдущей теоремы), и мы используем неперрывность меры снизу
				\end{enumerate}
		\end{enumerate}
\end{enumerate}			
	
\section{Линейность интеграла Лебега}

\end{document}