\documentclass[paper=a4, fontsize=15pt]{article}

\usepackage[russian]{babel}
\usepackage{scrextend}
\usepackage[utf8x]{inputenc}
\usepackage[T1,T2A]{fontenc}
\usepackage[left=1.5cm,right=1.5cm,top=1.5cm,bottom=1.5cm,bindingoffset=0cm]{geometry}
\usepackage[pdftex]{graphicx}
\usepackage{amsmath}
\usepackage{mathtools}
\usepackage{ulem}
\usepackage{mathrsfs}
\usepackage{amsfonts}
\usepackage{dsfont}
\usepackage{amssymb}
\usepackage{cmap}
\usepackage{hyperref}


\parindent=0cm

\title{Теоремы по матану, семестр 4}

\begin{document}
\maketitle
\tableofcontents
\newpage

\section{Характеризация измеримых функций с помощью ступенчатых (формулировка). Следствия}

\section{Измеримость монотонной функции}

\section{Теорема Лебега о сходимости почти везде и сходимости по мере}

\section{Теорема Рисса о сходимости по мере и сходимости почти везде}

\section{Простейшие свойства интеграла Лебега}
\subsection{Для определения (5)}
\begin{enumerate}
	\item $\int\limits_{\mathds{X}}f$ не зависит от представления $f$ как ступенчатой функции, то есть если $f$ реализуется как $f = \sum\limits_{k}(\lambda_k \cdot \chi_{E_k})$ и как $f = \sum\limits_{l}(\alpha_l \cdot \chi_{G_l})$, интегралы по этим функциям равны
	
	\proofname{:}
	
	Выпишем общее разбиение для этих двух разбиений
	
	Пусть $F_{ij} = E_i \cap G_j$
	
	Тогда $f = \sum\limits_{k}(\lambda_k \cdot \chi_{E_k}) = \sum\limits_{l}(\alpha_l \cdot \chi_{G_l}) = \sum\limits_{i, j}(\lambda_i (= \alpha_j) \cdot \chi_{F_{i, j}})$
	
	$\int f = \sum\limits_{i, j}(\lambda_i \cdot \mu F_{i, j}) = \sum\limits_i (\lambda_i \cdot \sum\limits_j (\mu F_{i, j})) = \sum\limits_i (\lambda_i \cdot \mu E_i) = \int f$ для первого разбиения
	
	Аналогично для второго разбиения получаем 
	
	$\int f = \sum\limits_{i, j}(\lambda_i \cdot \mu F_{i, j}) = \sum\limits_j (\alpha_i \cdot \sum\limits_i (\mu F_{i, j})) = \sum\limits_j (\lambda_j \cdot \mu G_i) = \int f$ для второго разбиения, что и требовалось доказать
	
	\item $f, g$ -измеримые ступенчатые функции, $f \leqslant g$, тогда $\int\limits_{\mathds{X}} f \leqslant \int\limits_{\mathds{X}} g$
	
	\proofname{:}
	
	Пусть $f = \sum\limits_{k}(\lambda_k \cdot \chi_{E_k})$, $g = \sum\limits_{l}(\alpha_l \cdot \chi_{G_l})$
	
	Аналогично доказательству предыдущей теоремы, строим общее ступенчатое разбиение
	
	Пусть $F_{ij} = E_i \cap G_j$
	
	Тогда $\int f = \sum\limits_{i, j}(\lambda_i \cdot \mu F_{i, j}) \leqslant \sum\limits_j(\alpha_j \cdot \mu F_{i, j}) = \int g$, что и требовалось доказать
\end{enumerate}

\subsection{Для определения (6)}

\section{Счетная аддитивность интеграла (по множеству)}
Тут будет вторая теорема

\section{Теорема Леви}

\section{Линейность интеграла Лебега}

\end{document}